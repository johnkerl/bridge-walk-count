\documentclass[12pt]{article}
\usepackage{verbatim}
\usepackage{amssymb,amsmath}
\usepackage{amsthm}
\usepackage{graphicx}
\usepackage{epsfig}
\begin{comment}
\setlength{\textwidth}{6.50in}
\setlength{\oddsidemargin}{0in}
\setlength{\evensidemargin}{0in}
\setlength{\textheight}{8.5in}
\setlength{\topmargin}{-.25in}
\end{comment}
%\newtheorem{theorem}{Theorem}
\newtheorem{corollary}{Corollary}[section]
\newtheorem{lemma}[corollary]{Lemma}
\newtheorem{proposition}[corollary]{Proposition}
\newtheorem{theorem}[corollary]{Theorem}
%\newtheorem{conjecture}[corollary]{Conjecture}
\newcommand{\Prob} {{\bf P}}
\newcommand{\Z}{{\mathbb Z}} 
\newcommand{\E}{{\bf E}}
\newcommand{\Es}{{\rm Es}}
\newcommand{\Cp}{{\rm cap}}
\newcommand{\R}{{\mathbb{R}}}
\newcommand{\C}{{\mathbb C}}
\newcommand{\rad}{{\rm rad}}
\newcommand{\hdim}{{\rm dim}_h}
\newcommand{\bound}{\partial_O}
\newcommand{\dist}{{\rm dist}}
\newcommand{\x}{{\bf x}}
\newcommand{\y}{{\bf y}}
\newcommand{\sphere}{{\cal S}}
\newcommand{\Var}{{\rm Var}}
\newcommand{\Q}{{\bf Q}}
\newcommand{\add}{+}
\newcommand{\bdim}{{\rm dim}_b}
\newcommand{\lf}{\lfloor}
\newcommand{\rf}{\rfloor}

\newcommand{\be}{\begin{equation}}
\newcommand{\ee}{\end{equation}}
\newcommand{\bea}{\begin{eqnarray}}
\newcommand{\eea}{\end{eqnarray}}
\newcommand{\beann}{\begin{eqnarray*}}
\newcommand{\eeann}{\end{eqnarray*}}

\def\reff#1{(\ref{#1})}

\def \dHalf  {{\cal H}}
\def \dhulls {{\cal A}}
\def \dhcap {{\rm dhcap}}
\def \Im {{\rm Im}}
\def \Re {{\rm Re}}
\def \p {\partial}
\def \edges {{\cal E}}
\def \Half {{\mathbb H}}
\def \reals {{\mathbb R}}
\def \interior {\rm int}
\def \LEW{LEW}
\def \sledim{\hat d}
\def \Disk {{\mathbb D}}
\def \Square {{\cal S}}
\def \diam {{\rm diam}}
\def \state {{\cal X}}
\def \eset {\emptyset}

\def \saws {{\cal R}}
\def \paths {{\cal K}}
\def \loop {{\cal L}}
\def \um  {{\bar m}}
\def \soup {{\cal C}}
\def \tree {{\cal T}}
\def \sgn {{\rm sgn}}
\def \looper {{\cal J}}
\def \strip{{\cal S}}

\def \scaleexp{{\rho}}

\def \energy {{\cal E}}
\def \hcap {{\rm hcap}}
\def \rect{{\cal R}}
\def \distsub {{\Upsilon}}
\def \distsubtwo {{\Theta}}

\def \dyad {{\cal D}}

\def \F {{\mathcal F}}

\def \walk {{\mathcal W}}
\def \hwalk {{\mathcal H}}
\def \bridge{{\mathcal B}}
\def \ibridge{{\mathcal I}}

\def \acat {{\oplus}}
\def \bcat {{\otimes}}


\def \Define {\noindent {\bf Definition.} }
\def  \Pf  {\noindent {\bf Proof.} }
\def \freemass {{\cal C}}
\newenvironment{remark}[1][Remark]{\begin{trivlist}
\item[\hskip \labelsep {\bfseries #1}]}{\end{trivlist}}
\newenvironment{definition}[1][Definition]{\begin{trivlist}
\item[\hskip \labelsep {\bfseries #1}]}{\end{trivlist}}
\newenvironment{example}[1][Example]{\begin{trivlist}
\item[\hskip \labelsep {\bfseries #1}]}{\end{trivlist}}
\newenvironment{conjecture}[1][Conjecture]{\begin{trivlist}
\item[\hskip \labelsep {\bfseries #1}]}{\end{trivlist}}

\begin{document}


\section{SLE partition functions}

For a domain $D$ and points $z,w$ on its boundary, we let 
$H_D(z,w)$ denote the total mass of the SAW's in $D$ from $z$ to $w$. 
This is a rather vague definition. Our goal is to argue that it 
should be conformally covariant in the following sense. If $\Phi$ is 
a conformal map on $D$, then 
\bea
H_D (z,w)  = 
[\Phi_A^\prime(z) \Phi_A^\prime(w)]^{5/8} 
H_{\Phi(D)}(\Phi(z),\Phi(w)) 
\label{eqzz}
\eea
This formula should not be taken too literally. In particular, there 
are lattice effects to be taken into account. 

Let 
\beann
Z(0,n) = \sum_{\omega:0 \rightarrow n, \omega \subset \Half} \beta^{|\omega|} 
\eeann
The sum is over all self-avoiding walks that start at $(0,0)$, end 
at $(n,0)$  and stay in the upper half plane except for their endpoints.
We assume that there is an exponent $\scaleexp$ such that 
the limit 
\beann
c=\lim_{n \rightarrow \infty} Z(0,n) n^\rho
\eeann
exists. Now let $x >0$. Let $[nx]$ denote the integer closest to $x$. 
We have
\beann
\lim_{n \rightarrow \infty} Z(0,[nx]) n^\scaleexp = 
\lim_{n \rightarrow \infty}  Z(0,[nx]) [nx]^\scaleexp 
{n^\scaleexp \over [nx]^\scaleexp}
= {c \over x^\scaleexp}
\eeann
This shows that 
\bea
H_\Half(0,x) = \frac{1}{x^\scaleexp} \, H_\Half(0,1)
\label{scaling}
\eea

Let $A \subset \Half$ be such that $\Half \setminus A$ is a simply connected
domain which is bounded away from $0$. The scaling limit of the 
self-avoiding walk is conjectured to be $SLE_{8/3}$, a 
chordal restriction measure.
The probability that a SAW $\gamma$ in $\Half$ from $0$ to $\infty$ does not 
enter $A$ is given by 
\beann
P(\gamma \cap A = \emptyset) = [\phi_A^\prime(0)]^{5/8}
\eeann
where $\phi_A$ is a conformal map from $\Half \setminus A$ to $\Half$
which sends $\infty$ to itself and has derivative $1$ at $\infty$. 
(This means $\phi(z) \asymp z$ as $z \rightarrow \infty$.)
If we assume that the scaling limit of the probability mesasure
for the SAW between two points is conformally invariant,
then we can find the probability that a SAW in $\Half$ from $0$ to a 
real $x$ hits $A$. Denote this probability by 
$P_{\Half,0,x}(\gamma \cap A =\emptyset)$.
Let $\psi_x$ be a Moibius transformation of 
$\Half$ to itself which fixes $0$ and sends $x$ to $\infty$. 
Then 
\beann
P_{\Half,0,x}(\gamma \cap A =\emptyset)=  [\phi_{\psi_x(A)}^\prime(0)]^{5/8}
\eeann
We can express this probability as a ratio of partition functions
\beann
P_{\Half,0,x}(\gamma \cap A =\emptyset) = {H_{\Half \setminus A} (0,x) 
\over H_\Half(0,x) }
\eeann
So we have the following transformation result for partition functions
\beann
H_{\Half \setminus A} (0,x)  = [\phi_{\psi_x(A)}^\prime(0)]^{5/8} H_\Half(0,x) 
\eeann

Recall that $\phi_{\psi_x(A)}$ is a conformal map that takes 
$\Half \setminus \psi_x(A)$ to $\Half$ and fixes $0$ and $\infty$. 
Letting
\beann
\phi_{x,A}= \psi_x^{-1} \circ \phi_{\psi_x(A)} \circ \psi_x,
\eeann
$\phi_{x,A}$ maps $\Half \setminus A$ to $\Half$ and fixes $0$ 
and $x$. 
Since $\psi_x(0)=0$, $\phi_{\psi_x(A)}(0)=0$, and 
$(\psi_x^{-1})^\prime(0)=1/\psi_x^\prime(0)$, 
we have that 
\beann
\phi_{x,A}^\prime(0) = \phi_{\psi(A)}^\prime(0) 
\eeann
So we have
\bea
H_{\Half \setminus A} (0,x)  = [\phi_{x,A}^\prime(0)]^{5/8} H_\Half(0,x) 
\label{trans}
\eea
Of course there is more that one conformal map of $\Half \setminus A$
to $\Half$ that fixes $0$ and $x$. The particular map we defined has 
$\phi_{x,A}^\prime(x)=1$. 
A general conformal map $\Phi_{x,A}$ of $\Half \setminus A$
to $\Half$ that fixes $0$ and $x$ can be written as 
$\Phi_{x,A} = \chi \circ \phi_{x,A}$ 
where $\chi$ is a Moibius transformation of $\Half$ to itself that 
fixes $0$ and $x$. 
Such transformations are given by 
\beann 
\chi(z) = {z \over c(z-x)+1}
\eeann
where $c \in \reals$. 
A short computation shows $\chi^\prime(0) \chi^\prime(x)=1$. 
Since 
\beann
\Phi_{x,A}^\prime(0) &=& \chi^\prime(0) \phi_{x,A}^\prime(0) \\
\Phi_{x,A}^\prime(x) &=& \chi^\prime(x) \phi_{x,A}^\prime(x) = \chi^\prime(x) 
\eeann
we find
\beann
\phi_{x,A}^\prime(0) = \Phi_{x,A}^\prime(0)  \Phi_{x,A}^\prime(x)  
\eeann
So \reff{trans} can be written as 
\bea
H_{\Half \setminus A} (0,x)  = 
[\Phi_{x,A}^\prime(0) \Phi_{x,A}^\prime(x)]^{5/8} H_\Half(0,x) 
\label{eqaa}
\eea
where $\Phi_{x,A}$ is any conformal map of $\Half \setminus A$ to $\Half$ 
which fixes $0$ and $x$.


Now let $\Phi_A$ be a conformal map that maps $\Half \setminus A$
onto $\Half$ and fixes $0$ but not $x$. Let $\lambda=\Phi_A(x)/x$. 
Then $\Phi_{x,A} = \lambda^{-1} \Phi_A$ fixes $x$, and so \reff{eqaa}
implies
\beann
H_{\Half \setminus A} (0,x)  = 
[\Phi_A^\prime(0) \Phi_A^\prime(x)]^{5/8} \lambda^{-5/4} H_\Half(0,x) 
\label{eqbb}
\eeann
Using \reff{scaling} we can rewrite this as 
\bea
H_{\Half \setminus A} (0,x)  = 
[\Phi_A^\prime(0) \Phi_A^\prime(x)]^{5/8} \lambda^{\rho-5/4} 
H_\Half(0,\Phi_A(x)) 
\eea

Until now we have been assuming the conformal map $\Phi_A$ fixes $0$. 
By adding a real constant to the map we can relax this constraint. 
Since $H_\Half(x,y) = H_\Half(y-x)$, we conclude that for any 
conformal map of $\Half \setminus A$ to $\Half$, 
\bea
H_{\Half \setminus A} (0,x)  = 
[\Phi_A^\prime(0) \Phi_A^\prime(x)]^{5/8} \lambda^{\rho-5/4} 
H_\Half(\Phi_A(0),\Phi_A(x)) 
\eea
If $\rho=5/4$ this estabilishes \reff{eqzz} for a limited 
class of domains. 


\end{document}
\end